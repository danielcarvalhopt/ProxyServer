\documentclass{llncs}
\usepackage{times}
\usepackage[T1]{fontenc}

% Comentar para not MAC Users
\usepackage[utf8x]{inputenc}

\usepackage{a4}
%\usepackage[margin=3cm,nohead]{geometry}
\usepackage{epstopdf}
\usepackage{graphicx}
\usepackage{fancyvrb}
\usepackage{amsmath}
%\renewcommand{\baselinestretch}{1.5}

\begin{document}
\mainmatter
\title{Web Proxy Cache Server}

\titlerunning{Web Proxy Cache Server}

\author{Daniel Carvalho e Ricardo Branco}

\authorrunning{Daniel Carvalho \and Ricardo Branco}

\institute{
University of Minho, Department of  Informatics, 4710-057 Braga, Portugal\\
e-mail: \{a61008,a61075\}@alunos.uminho.pt
}

\date{\today}
\bibliographystyle{splncs}

\maketitle
\begin{abstract}
  Este artigo reflecte a implementação e a descrição do funcionamento de um servidor de \textit{"proxy"} com cache na 
  linguagem de programação Java. Nesta implementação recorre-se ao uso de 
  "threads" de forma a ser possível tratar cada pedido independentemente mantendo 
  controlo sobre a concorrência entre pedidos ao servidor e sobre o armazenamento na 
  cache. Para além disso o servidor suporta também outro tipo de objectos que 
  não somente páginas HTML.
\end{abstract}

\section{Introdução}

No âmbito da unidade curricular de Redes de Computadores foi proposta a 
implementação de um servidor de "proxy" com cache, com suporte a vários 
utilizadores em Java que suportasse vários objectos que não só HTML. Neste nível 
tornam-se importantes os conhecimentos obtidos tanto na própria unidade curricular 
como na de Sistemas distribuídos onde a aprendizagem passa também por 
manipulação de software cliente-servidor na sua vertente de computação 
distribuída. Após a análise do código disponibilizado, este foi completo e 
melhorado de forma a proporcionar o funcionamento desejado.

\section{Completamento de código ("Fill-in")}
Na sua maioria a necessidade de completar o código passava pela análise de 
variáveis de manipulação de dados já definidas no código fornecido e pela interpretação e criação de 
variáveis Socket e SocketServer para tornar possível a criação de ligações entre 
o cliente e o servidor de "proxy" e entre o servidor de "proxy" e o servidor que consta no pedido do cliente.
Tornou-se também necessária a
análise e interpretação dos pedidos GET feitos pelo cliente e das respostas dos 
servidores. Por fim e num nível mais avançado foram implementadas as 
funcionalidades de cache e threads que se falarão nas secções seguintes.
Exemplos de completamento:
\begin{verbatim}
      socket = /* Fill in */;
      // After
      socket = new ServerSocket(port);
      
      request = /* Fill in */
      // After
      request = new HttpRequest(fromClient);
      
      DataInputStream fromServer = /* Fill in */;
      response = /* Fill in */;
      DataOutputStream toClient = /* Fill in */;
      /* Fill in */
      // After
      DataInputStream fromServer = 
          new DataInputStream(server.getInputStream()); 
      response = new HttpResponse(fromServer); 
      DataOutputStream toClient = 
          new DataOutputStream(client.getOutputStream());
      ProxyCache.caching(request, response); 
      toClient.writeBytes(response.toString()); 
      toClient.write(response.body); 

\end{verbatim}
\subsection{Cache}
No que diz respeito à implementação de um sistema de cache no servidor, recorreu-se 
ao uso de estruturas de dados do Java e a ficheiros de forma a fazer 
corresponder cada URI específico ao seu ficheiro de cache criado após ser 
estabelecida uma ligação entre o servidor designado pelo dito URI e o servidor 
de "proxy". Assim sendo, é criada uma directoria designada por Cache na 
directoria de execução do software servidor onde são armazenados os ficheiros 
obtidos ao fazer "caching" dos servidores constantes nos pedidos dos clientes.
A estrutura de dados que permite a cada URI fazer corresponder o seu ficheiro em 
cache é a HashTable, os motivos do seu uso serão explicados na secção seguinte.
Código demonstrativo:
\begin{verbatim}
private static Map<String, String> cache = new Hashtable<String, String>();
public synchronized static void caching
            (HttpRequest pedido, HttpResponse resposta) throws IOException{
    	File ficheiro;
    	DataOutputStream paraficheiro;
    	
    	ficheiro = new File("cache/","cached_"+System.currentTimeMillis());
    	paraficheiro = new DataOutputStream( new FileOutputStream(ficheiro));
    	paraficheiro.writeBytes(resposta.toString()); /* Escreve headers */
    	paraficheiro.write(resposta.body); /* Escreve body */
    	paraficheiro.close();
    	cache.put(pedido.URI, ficheiro.getAbsolutePath());
    	System.out.println("Caching from: "+pedido.URI+
    	        " para "+ficheiro.getAbsolutePath());
    }
public synchronized static byte[] uncaching
        (String uripedido)   throws IOException{
  		File ficheirocached;
  		FileInputStream deficheiro;
  		String hashfile;
  		byte[] bytescached;

  		if((hashfile = cache.get(uripedido))!=null){
  			ficheirocached = new File(hashfile);
  			deficheiro = new FileInputStream(ficheirocached);
  			bytescached = new byte[(int)ficheirocached.length()];
  			deficheiro.read(bytescached);
  			System.out.println("Caching: Hit on "+uripedido
  			      +" returning cache to user");
  			return bytescached;
  		}
  		else {
  			System.out.println("Caching: No hit on "+uripedido);
  			return bytescached = new byte[0];
  		}
}
\end{verbatim}
\subsection{Threads}
De forma a suportar os vários pedidos de cliente e conseguir controlar a 
concorrência entre eles recorreu-se ao uso de Threads. Assim sendo cada thread 
trata de processar o pedido de cada cliente associado bem como a resposta a esse 
cliente. Assim sendo também o "caching" de páginas é feito por cada thread assim 
como a verificação de "cache hits" são feitos por cada thread individualmente. 
Tendo em conta o uso de threads optou-se por usar a estrutra de dados do Java, 
Hashtable que já é por definição "synchronized" ou seja, preparada para 
assegurar o controlo de concorrência na sua própria implementação. Não obstante 
também os métodos de "caching" e "uncaching" são "synchronized". 
\begin{verbatim}
// Classe ProxyCache
  client = socket.accept();
  (new Thread(new Threads(client))).start(); 
  
// Classe Threads
public class Threads implements Runnable{
	private final Socket client;

    public Threads(Socket client) {
        this.client = client;
    }	
    
    public void run() {
	Socket server = null;
	HttpRequest request = null;
	HttpResponse response = null;
    
    // Código de tratamento de client requests e server responses
    }
}
  \end{verbatim}



\section{Conclusões}
Neste projecto foi necessário tomar algumas decisões já descritas e explicadas 
durante o artigo, sendo as mais importantes relacionadas com o funcionamento e 
armazenamento da cache assim como o uso de threads no controlo das ligações dos 
clientes ao servidor para tratamento dos HTTP requests e responses. Foi possível 
ver o funcionamento ao nível protocolar do HHTP, como a forma como os pedidos 
são feitos assim como as respostas, bem como a interacção entre várias máquinas 
mediadas por esse protocolo.


\end{document}